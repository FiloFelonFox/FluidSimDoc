\documentclass{article}

% Language setting
% Replace `english' with e.g. `spanish' to change the document language
\usepackage[polish]{babel}

% Set page size and margins
% Replace `letterpaper' with `a4paper' for UK/EU standard size
\usepackage[letterpaper,top=2cm,bottom=2cm,left=3cm,right=3cm,marginparwidth=1.75cm]{geometry}

% Useful packages
\usepackage{amsmath}
\usepackage{graphicx}
\usepackage[colorlinks=true, allcolors=blue]{hyperref}

\usepackage{algorithm}
\usepackage{algorithmic}

\title{Praca inżynierska}
\author{ehhhhhhhhh}

\begin{document}
\maketitle
\newpage
\tableofcontents
\newpage

\section{Wstęp}
Obliczenia numeryczne dotyczące symulacji płynów odgrywają kluczową rolę we współczesnej inżynierii i naukach techninych. Opiera się na nich wiele dziedzin oraz sektorów branżowych takich jak: inżynieria chemiczna, energetyka, lotnictwo, motoryzacja, meteorologia, a także przemysły growy i rozrywkowy. Dzięki nim możliwa jest uzyskanie zadawalających rezultatów przewidywania zachowania cieczy bez konieczności przeprowadzania czasochłonnych i skomplikowanych obliczeń bądź odwzorowywania układów o dużej złożoności.\newline\newline
Wzrost dokładności modeli fizycznych oraz zapotrzebowań na rozdzielczość i wyprzedzenie dokonywanych obliczeń powoduje wzrost wymaganej mocy obliczeniowej i pamięci. Złożoność obliczeniowa problemu symulacji powoduje że istotnym ograniczeniem może być czas jej wykonania. Konsekwentnie wysokiego znaczenia nabierają metody optymalizacji, których celem jest ograniczenie wykorzystywanej pamięci oraz czasu koniecznego do przeprowadzenia obliczeń.\newline

\section{Cel i zakres}
Celem niniejszej pracy inżynierskiej jest zaproponowanie i analiza rozwiązań dotyczących optymalizacji symulacji płynów. Praca koncentruje się na rozwiązaniach dotyczących metod wyszukiwania cząstek sąsiadujących w metodzie cząstek wygładzonych (SPH) opartej o równania Naviera-Stokesa.\newline\newline
Zakres pracy obejmuje omówienie podstaw teoretycznych dotyczących symulacji płynów, zaproponowanie metod optymalizacji i ich analizę teoretyczną, oraz przedstawienie wyników zastosowania tychże metod w praktyce. Uzyskane rezultaty mogą stanowić podstawę do dalszych badań na temat optymalizacji oraz praktycznego wykorzystania wypracowanych rozwiązań.\newline

\section{Wstęp teoretyczny}
\subsection{Równania Naviera-Stokesa}
Równania Naviera–Stokesa stanowią podstawowy zestaw równań opisujących ruch płynów, zarówno cieczy, jak i gazów. Ich sformułowanie opiera się na zasadzie zachowania pędu, zgodnie z którą zmiany pędu elementarnej objętości płynu wynikają z oddziaływania sił działających na ten element. Równania zakładają że zmiany zależą od sił lepkości, sił wynikających z ciśnienia oraz sił masowe, takich jak siła grawitacji.\newline

Równania zapisane są w postaci równań różniczkowych cząstkowych i do ich rozwiązania niezbędne jest zastosowanie narzędzi rachunku różniczkowego i całkowego. Przez wzgląd na swoją nieliniowość oraz sprzężenie pomiędzy składowymi prędkości, ich rozwiązanie stanowi istotne wyzwanie obliczeniowe, co powoduje, że uzyskanie rozwiązań drogą analityczną jest realistyczne jedynie w najprostszych przypadkach, takich jak ustalony, laminarny i nieturbulentny przepływ w geometriach prostych.\newline

W przypadkach bardziej złożonych zagadnień, w szczególności przepływów niestacjonarnych, trójwymiarowych lub turbulentnych, stosowanych m.in. w obliczeniach aerodynamicznych i inżynierskich, konieczne jest wykorzystanie metod numerycznych. Metody te, implementowane w narzędziach obliczeniowej mechaniki płynów (CFD), umożliwiają przybliżone rozwiązanie równań Naviera–Stokesa i analizę rzeczywistych zjawisk przepływowych, które nie są możliwe do opisania metodami analitycznymi.\newline
Zapis równań Naviera-Stokesa w kartezjańskim układzie współrzędnych wygląda następująco:

$$ 
\rho 
\frac{du }{dt }
= X 
- \frac{\partial  p }{\partial x }
+ \frac \partial {\partial  x }
\left [ 
    \mu \left ( 
        2 \frac{\partial u }{\partial  x }
        - { \frac{2}{3} }  \mathop{\rm div}  \mathbf w 
    \right )  
\right ] 
+
$$

$$ 
+ 
\frac \partial {\partial  y }
\left [ 
    \mu \left ( 
        \frac{\partial u }{\partial  y }
        + \frac{\partial v }{\partial  x }
    \right )  
\right ]
+ 
\frac \partial {\partial  z }
\left [ 
    \mu \left ( 
        \frac{\partial w }{\partial  x }
        + \frac{\partial u }{\partial  z }
    \right )  
\right ],
$$

$$ 
\rho 
\frac{dv }{dt }
= Y
- \frac{\partial  p }{\partial y }
+ \frac \partial {\partial  y }
\left [ 
    \mu \left ( 
        2 \frac{\partial v }{\partial  y }
        - { \frac{2}{3} }  \mathop{\rm div}  \mathbf w 
    \right )  
\right ] 
+
$$

$$ 
+ 
\frac \partial {\partial  z }
\left [ 
    \mu \left ( 
        \frac{\partial v }{\partial  z }
        + \frac{\partial w }{\partial  y }
    \right )  
\right ]
+ 
\frac \partial {\partial  x }
\left [ 
    \mu \left ( 
        \frac{\partial u }{\partial  y }
        + \frac{\partial v }{\partial  x }
    \right )  
\right ],
$$

$$ 
\rho 
\frac{dw }{dt }
= Z 
- \frac{\partial  p }{\partial z }
+ \frac \partial {\partial  z }
\left [ 
    \mu \left ( 
        2 \frac{\partial w }{\partial  z }
        - { \frac{2}{3} }  \mathop{\rm div}  \mathbf w 
    \right )  
\right ] 
+
$$

$$ 
+ 
\frac \partial {\partial  x }
\left [ 
    \mu \left ( 
        \frac{\partial w }{\partial  x }
        + \frac{\partial u }{\partial  z }
    \right )  
\right ]
+ 
\frac \partial {\partial  y }
\left [ 
    \mu \left ( 
        \frac{\partial v }{\partial  z }
        + \frac{\partial w }{\partial  y }
    \right )  
\right ],
$$

$$ 
\frac{\partial \rho }{\partial t}
+ \frac{\partial (\rho u) }{\partial x}
+ \frac{\partial (\rho v) }{\partial y}
+ \frac{\partial (\rho w) }{\partial z}
= 0,
$$

gdzie:\newline
w - wektor prędkości o rzutach u, v, w na osie współrzędnych x, y, z,\newline
p - ciśnienie,\newline
$\rho$ - gęstość,\newline
$\mu$ - współczynnik lepkości,\newline
X,Y,Z - rzuty wektora sił masowych K na osie współrzędnych,\newline
(pochodna substancjonalna) - pochodna substancjonalna,\newline

Równania są wyprowadzone na podstawie Newtonowskiego uogólnionego prawa tarcia.
Analiza płynów ściśliwych wymaga dodatkowo zastosowania równań uwzględniających ciśnienie, gęstość, temperaturę i przepływy energii.
Dla nieściśliwych izotermicznych płynów, równania przyjmują następującą formę:
$$
\rho \frac {dw}{dt}
= \boldsymbol{K} - grad \space \rho
+ \mu \Delta \boldsymbol{w},
$$
$$
div \space \boldsymbol{w} = 0.
$$


\subsection{Metoda cząsteczkowa symulacji}
\subsubsection{Równania Naviera-Stokesa}
Przyjmijmy następującą formę równań dla cieczy nieściśliwych:\newline
$$
\rho \left (
    \frac{\partial v}{\partial t} + v \nabla v
\right )
= - \nabla p + \rho f_{zew} + \mu \nabla^2v,
$$
$$
\frac{\partial \rho}{\partial t} + \nabla(\rho v) = 0
$$

Wykorzystanie metody cząsteczkowej pozwala na pewne uproszczenia, mianowicie:\newline
(1) Stała liczba cząsteczek zapewnia zachowanie masy wewnątrz systemu\newline
(2) wyrażenie (wstaw wyrażenie) może zostać zastąpione operatorem Stokesa(wstaw drugie). Ponieważ cząstki poruszają się wraz z płynem operator Stokesa pola prędkości jest pochodną czasu prędkości co znaczy, że wyrażenie $v \nabla v$ nie jest potrzebne.\newline
Oznacza to że równanie przyjmuje następującą formę:\newline
$$
\rho \frac{D v}{D t}
= - \nabla p + \rho f_{zew} + \mu \nabla^2v,
$$
Prawa strona równania składa się z trzech składowych:\newline
$- \nabla p$- odwrotność gradientu ciśnienia\newline
Odpowiada za siły wynikające z miejscowych różnic w ciśnieniu wewnątrz płynu.\newline
Współczynnik ten odpowiada za ruch miejscowy płynu podyktowany gradientem ciśnienia, powodując dążenie do wyrównania ciśnienia w ośrodku.\newline
$\rho f_{zew}$ - siły zewnętrzne\newline
Odpowiadają siłom jednorodnym w symulowanej przestrzeni. Dotyczą one sił zewnętrznych jak grawitacja, a w przypadku symulacji objętości płynów poddawanych innym przyśpieszeniom wynikającym na przykład z ruchu ośrodka, również takowym.\newline
Ponieważ zastosowana wersja symulacji nie zakłada sił innych niż siła grawitacji równanie wygląda następująco:\newline
$$
\rho \frac{D v}{D t}
= - \nabla p + \rho g + \mu \nabla^2v,
$$
$\mu \nabla^2v$ - siły lepkości\newline
Siły powodujące wygładzenie pola prędkości płynu spowodowane jego lepkością.\newline


\subsubsection{Metoda cząstek wygładzonych}
Metoda cząstek wygładzonych (ang. Smoothed Particle Hydrophysics - SPH) to bezsiatkowa metoda numeryczna stosowana do symulacji przepływów płynów i innych zjawisk ciągłych. Pierwotnie opracowana do symulacji zjawisk astrofizycznych aktualnie znajduje zastsowanie m.in. w mechanice płynów, inżynierii materiałowej, czy grafice komputerowej. Brak zastosowania siatki obliczeniowej jest dużą zaletą tej metody, pozwalającą na modelowanie znacznych deformacji, swobodnych powierzchni i zjawisk nieciągłych.\newline\newline
W metodzie tej ośrodek ciągły jest reprezentowany jako zbiór dyskretnych cząstek, które zawierają informację o lokalnych wartościach pól takich jak masa, gęstość, prędkość itp. Każda z cząstek oddziałowuje jedynie z cząstkami które znajdują się w jej bezpośrednim sąsiedztwie określonym przez promień wygładzania $h$. Wartości pól w danym punkcie przestrzeni są aproksymowane za pomocą funkcji wygładzającej, która dyktuje wagę z jaką uwzględniany jest wkład cząstek sąsiadujących w zależności od odległości.\newline\newline
Podstawą metody jest przybliżenie całki opisującej wielkość fizyczną w postaci sumy po cząstkach:\newline
$$
A(r) \approx \sum_{j}A_j \frac{m_j}{\rho_j}W(r-r_j,h),
$$
Gdzie:\newline
A - dowolna wielkość fizyczna,\newline
m - masa cząstki,\newline
$\rho$ - gęstość w punkcie cząstki,\newline
W - funkcja wygładzająca\newline
r - odległość od cząstki,\newline
h - odległość oddziaływania,\newline\newline
Równania opisujące zmiany stanu cząstek, takie jak równania Naviera-Stokesa są w metodzie SPH zapisywane w postaci dyskretnej, w oparciu o tą właśnie metodę obliczania wartości fizycznej w punkcie.\newline
Do zalet metody SPH należy prostota jej implementacji, łatwość zrównoleglenia oraz możliwość modelowania zjawisk z ruchomymi granicami.\newline
Głównymi wadami są natomiast wysoka złożoność obliczeniowa spowodowana koniecznością wyszukiwania cząstek sąsiadujących i problemy wynikające z trudności zachowania stabilności numerycznej oraz możliwa niska dokładność przy nieodpowiednich parametrach.\newline
Dlatego też metoda SPH wymaga zastosowania technik optymalizacyjnych i doboru odpowiednich parametrów symulacji.\newline


\subsubsection{Ciśnienie}
Zastosowanie metody SPH dla ciśnienia opisanego jako wyrażenie $- \nabla p$:
$$
f_i = -\sum_j m_j \frac{p_j}{\rho_j}\nabla W(r_i - r_j, h),
$$
Wstępna analiza równania pozwala zauważyć, że dla dwóch oddziałujących cząstek siła ta nie jest symetryczna. Jest to spowodowane możliwością wystąpienia różnicy ciśnień w miejscach dwóch cząstek. Istnieje wiele metod wyrównania tych sił. Jedną z nich jest proste uśrednienie wartości ciśnienia tychże cząstek:\newline
$$
f_i = -\sum_j m_j \frac{p_j + p_i}{2\rho_j}\nabla W(r_i - r_j, h),
$$
Ciśnienie p wyznaczane jest równaniem gazu idealnego:\newline
$$
p=k\rho
$$
gdzie k to stała gazu zależna od temperatury.\newline
Desburn sugeruje wykorzystanie zmodyfikowanej wersji równania:\newline
$$
p=k(\rho - \rho_0)
$$
gdzie $\rho_0$ jest gęstością spoczynku.\newline
Pozwala to na wyznaczenie gęstości do której dążyć będzie symulowany płyn jako łatwo modyfikowalny parametr.\newline
Ponieważ równania zależą od gradientu ciśnienia, przesunięcie nie wpływa na obliczenia siły.

\subsubsection{Siły zewnętrzne}
Obliczenia dla siły grawitacji ograniczają się do prostego równania:\newline
$$
f_i = g
$$
\subsubsection{Lepkość}
Wyrażenie dla lepkości $\mu \nabla^2v$ przekształcone do sumy:\newline
$$
f_i = \mu\sum_j m_j \frac{v_j}{\rho_j}\nabla^2W(r_i-r_j,h)
$$
Jak widać również skutkuje siłami asymetrycznymi. Ponieważ siły te operują na różnicach prędkości, naturalnym jest usymetryzowanie ich na różnicach prędkości:\newline
$$
f_i = \mu\sum_j m_j \frac{v_j-v_i}{\rho_j}\nabla^2W(r_i-r_j,h)
$$

\subsubsection{Funkcje wygładzania}
Metoda wygładzonych cząstek opiera się na funkcjach wygładzania. Są to funkcje określające wzajemny wpływ cząstek w zależności od odległości.\newline
Funkcje wygładzania można traktować jako funkcje jednej zmiennej 
$$
W(r) = 
\begin{cases}
f(r,h), & dla \space r<h,\\
0,    & dla \space r>=h
\end{cases}
$$
gdzie:\newline
r - odległość analizowanej cząstki od cząstki sąsiadującej\newline
h - maksymalna odległość interakcji\newline
f - funkcja kształtu krzywej\newline
Jest to forma zakładająca stałą maksymalną odległość interakcji, co powoduje ograniczenia w dowolności konfiguracji symulacji.\newline
Alternatywną i zazwyczaj stosowaną formą jest wyrażenie funkcji wygładzania jako funkcji dwóch zmiennych:\newline
$$
W(r,h) = 
\begin{cases}
m(h)f(r,h), & dla \space r<h,\\
0,    & dla \space r>=h
\end{cases}
$$
Gdzie m to funkcja maksymalnej odległości interakcji, która służy jako współczynnik standaryzujący. Jej celem jest zachowanie stałej wartości całki funkcj wygładzającej. W przypadku jej niezastosowania zwiększenie odległości interakcji celem wygładzenia wartości pola spowoduje gwałtowny wzrost wartości. Istnienie współczynnika pozwala na swobodniejszą manipulację odległością interakcji, bez niepożądanych zachowań.\newline
Funkcja m jest określona poniższym wzorem:\newline
$$
\frac {\pi h \int_0^h{f(r,h)}}{m(h)} = const
$$

Dodatkowo w metodzie SPH wykorzystujemy pochodne funkcj wygładzających $\frac {dW(r,h)}{dr}$ celem wyznaczenia gradientów pól cech fizycznych.\newline

\section{Metody odnajdywania cząstek}
\subsection{Odnajdywanie siłowe}
Najprostszym sposobem odnajdywania cząstek sąsiadujących jest użycie metody siłowej (eng. brute-force). Metoda polega na przeszukaniu zbioru cząstek i sprawdzenia odległości od cząstki aktualnie analizowanej.\newline
Oznacza to że dla każdej możliwej pary cząstek konieczne jest wykonanie prównania odległości z maksymalną odległością interakcji:\newline
$$
||r_i-r_j|| \leq h.
$$

Jeśli warunek zostanie spełniony to znaczy, że cząstka j jest w sąsiedztwie czątki i i zostaje uwzględniona w dalszych obliczeniach.\newline

Postać algorytmiczna wyszukiwania metodą siłową:\newline

\begin{algorithm}
\caption{Metoda siłowa}
\begin{algorithmic}[1]
void CalculateViscosity(uint3 id : SV_DispatchThreadID)\\
{\\
	if (id.x >= numParticles)\\
		return;\\
        for (uint secondParticleIndex = 0;\\ secondParticleIndex < numParticles; secondParticleIndex++)\\
	{\\
			if (secondParticleIndex == id.x)\\
				continue;\\
			if (sqrDstToSecondParticle > sqrSmoothingRadius)\\
				continue;\\
            wykonaj_obliczenia();\\
    }\\
}
\end{algorithmic}
\end{algorithm}


Złożoność obliczeniowa metody siłowej odnajdywania czątek wynosi $O(N^2)$, co oznacza kwadratowy wzrost liczby operacji w stosunku do liczby cząstek.\newline
Metoda ta jest łatwa w implementacji oraz nie wymaga dodatkowych struktur, przez co może zostać wykorzystana w stosunkowo niewielkich symulacjach służących do demonstracji lub nauki.\newline
W zastosowaniach praktycznych metody SPH, gdzie liczba cząstek sięga tysiący, bądź milionów, siłowe odnajdywanie jest niewystarczające przez wzgląd na znaczący czas obliczeń.\newline


\subsection{Siatka jednorodna}
Jednym z rozwiązań mających na celu przyśpieszenie procesu wyszukiwania cząstek sąsiadujących jest zastosowanie siatki jednorodnej (eng. uniform grid).\newline
Metoda ta polega na podziale przestrzeni na regularne komórki o stałym rozmiarze. Rozmiar komórki dostosowywany jest do odległości interakcji h. Uzależnienie to pozwala ustalić stałą, ograniczoną liczbę komórek które należy przeszukać.\newline
Każdej cząstce przypisywana jest komórka siatki na podstawie jej pozycji:\newline
$$
i_n = \frac {x_n} r
$$
Gdzie:\newline
$i_n$ - indeks cząstki w wymiarze $n$\newline
$x_n$ - położenie cząstki wzdłuż osi $n$\newline
$r$ - wielkość komórki siatki\newline\newline
Na podstawie danego wzoru algorytm przypisywania komórki wygląda następująco:\newline
[algorytm przypisywania komórki]\newline
W takim przypadku liczba operacji koniecznych do stworzenia struktury wynosi:\newline
[wzór na liczbę operacji]\newline
Najczęściej wartość r jest równa, lub nieznacznie większa od odległości interakcji co ogranicza ilość koniecznycj do sprawdzenia komórek wzorem:\newline
$$
n_s = 3^d
$$
Gdzie:\newline
$n_s$ - liczba komórek koniecznych do sprawdzenia\newline
$d$ - liczba wymiarów symulacji\newline

Zastosowanie siatki pozwala uzależnić złożoność obliczeniową algorytmu od liczby cząstek oczekiwanej wewnątrz sąsiadujących komórek siatki, zamiast od całkowitej liczby cząstek.\newline

\begin{algorithm}
\caption{Wykorzystanie siatki}
\begin{algorithmic}[1]
\end{algorithmic}
\end{algorithm}

\subsection{Haszowanie przestrzenne}
\subsection{Stały indeks}
\subsection{Metody drzewiaste}

\section{Obliczenia wydajności}
\section{Symulacja}
\section{Wyniki badań}
\section{Podsumowanie}
\section{Wnioski}
\section{Bibliografia}

\end{document}
