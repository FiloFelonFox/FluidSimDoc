\documentclass{article}

% Language setting
% Replace `english' with e.g. `spanish' to change the document language
\usepackage[polish]{babel}

% Set page size and margins
% Replace `letterpaper' with `a4paper' for UK/EU standard size
\usepackage[letterpaper,top=2cm,bottom=2cm,left=3cm,right=3cm,marginparwidth=1.75cm]{geometry}

% Useful packages
\usepackage{amsmath}
\usepackage{graphicx}
\usepackage[colorlinks=true, allcolors=blue]{hyperref}

\title{Praca inżynierska}
\author{ehhhhhhhhh}

\begin{document}
\maketitle
\newpage
\tableofcontents
\newpage

\section{Wstęp}
Obliczenia numeryczne dotyczące symulacji płynów odgrywają kluczową rolę we współczesnej inżynierii i naukach techninych. Opiera się na nich wiele dziedzin oraz sektorów branżowych takich jak: inżynieria chemiczna, energetyka, lotnictwo, motoryzacja, meteorologia, a także przemysły growy i rozrywkowy. Dzięki nim możliwa jest uzyskanie zadawalających rezultatów przewidywania zachowania cieczy bez konieczności przeprowadzania czasochłonnych i skomplikowanych obliczeń bądź odwzorowywania układów o dużej złożoności.\newline\newline
Wzrost dokładności modeli fizycznych oraz zapotrzebowań na rozdzielczość i wyprzedzenie dokonywanych obliczeń powoduje wzrost wymaganej mocy obliczeniowej i pamięci. Złożoność obliczeniowa problemu symulacji powoduje że istotnym ograniczeniem może być czas jej wykonania. Konsekwentnie wysokiego znaczenia nabierają metody optymalizacji, których celem jest ograniczenie wykorzystywanej pamięci oraz czasu koniecznego do przeprowadzenia obliczeń.\newline

\section{Cel i zakres}
Celem niniejszej pracy inżynierskiej jest zaproponowanie i analiza rozwiązań dotyczących optymalizacji symulacji płynów. Praca koncentruje się na rozwiązaniach dotyczących metod wyszukiwania cząstek sąsiadujących w metodzie cząstek wygładzonych (SPH) opartej o równania Naviera-Stokesa.\newline\newline
Zakres pracy obejmuje omówienie podstaw teoretycznych dotyczących symulacji płynów, zaproponowanie metod optymalizacji i ich analizę teoretyczną, oraz przedstawienie wyników zastosowania tychże metod w praktyce. Uzyskane rezultaty mogą stanowić podstawę do dalszych badań na temat optymalizacji oraz praktycznego wykorzystania wypracowanych rozwiązań.\newline

\section{Wstęp teoretyczny}
\subsection{Równania Naviera-Stokesa}
Równania Naviera–Stokesa stanowią podstawowy zestaw równań opisujących ruch płynów, zarówno cieczy, jak i gazów. Ich sformułowanie opiera się na zasadzie zachowania pędu, zgodnie z którą zmiany pędu elementarnej objętości płynu wynikają z oddziaływania sił działających na ten element. Równania zakładają że zmiany zależą od sił lepkości, sił wynikających z ciśnienia oraz sił masowe, takich jak siła grawitacji.\newline

Równania zapisane są w postaci równań różniczkowych cząstkowych i do ich rozwiązania niezbędne jest zastosowanie narzędzi rachunku różniczkowego i całkowego. Przez wzgląd na swoją nieliniowość oraz sprzężenie pomiędzy składowymi prędkości, ich rozwiązanie stanowi istotne wyzwanie obliczeniowe, co powoduje, że uzyskanie rozwiązań drogą analityczną jest realistyczne jedynie w najprostszych przypadkach, takich jak ustalony, laminarny i nieturbulentny przepływ w geometriach prostych.\newline

W przypadkach bardziej złożonych zagadnień, w szczególności przepływów niestacjonarnych, trójwymiarowych lub turbulentnych, stosowanych m.in. w obliczeniach aerodynamicznych i inżynierskich, konieczne jest wykorzystanie metod numerycznych. Metody te, implementowane w narzędziach obliczeniowej mechaniki płynów (CFD), umożliwiają przybliżone rozwiązanie równań Naviera–Stokesa i analizę rzeczywistych zjawisk przepływowych, które nie są możliwe do opisania metodami analitycznymi.\newline
Zapis równań Naviera-Stokesa w kartezjańskim układzie współrzędnych wygląda następująco:

$$ 
\rho 
\frac{du }{dt }
= X 
- \frac{\partial  p }{\partial x }
+ \frac \partial {\partial  x }
\left [ 
    \mu \left ( 
        2 \frac{\partial u }{\partial  x }
        - { \frac{2}{3} }  \mathop{\rm div}  \mathbf w 
    \right )  
\right ] 
+
$$

$$ 
+ 
\frac \partial {\partial  y }
\left [ 
    \mu \left ( 
        \frac{\partial u }{\partial  y }
        + \frac{\partial v }{\partial  x }
    \right )  
\right ]
+ 
\frac \partial {\partial  z }
\left [ 
    \mu \left ( 
        \frac{\partial w }{\partial  x }
        + \frac{\partial u }{\partial  z }
    \right )  
\right ],
$$

$$ 
\rho 
\frac{dv }{dt }
= Y
- \frac{\partial  p }{\partial y }
+ \frac \partial {\partial  y }
\left [ 
    \mu \left ( 
        2 \frac{\partial v }{\partial  y }
        - { \frac{2}{3} }  \mathop{\rm div}  \mathbf w 
    \right )  
\right ] 
+
$$

$$ 
+ 
\frac \partial {\partial  z }
\left [ 
    \mu \left ( 
        \frac{\partial v }{\partial  z }
        + \frac{\partial w }{\partial  y }
    \right )  
\right ]
+ 
\frac \partial {\partial  x }
\left [ 
    \mu \left ( 
        \frac{\partial u }{\partial  y }
        + \frac{\partial v }{\partial  x }
    \right )  
\right ],
$$

$$ 
\rho 
\frac{dw }{dt }
= Z 
- \frac{\partial  p }{\partial z }
+ \frac \partial {\partial  z }
\left [ 
    \mu \left ( 
        2 \frac{\partial w }{\partial  z }
        - { \frac{2}{3} }  \mathop{\rm div}  \mathbf w 
    \right )  
\right ] 
+
$$

$$ 
+ 
\frac \partial {\partial  x }
\left [ 
    \mu \left ( 
        \frac{\partial w }{\partial  x }
        + \frac{\partial u }{\partial  z }
    \right )  
\right ]
+ 
\frac \partial {\partial  y }
\left [ 
    \mu \left ( 
        \frac{\partial v }{\partial  z }
        + \frac{\partial w }{\partial  y }
    \right )  
\right ],
$$

$$ 
\frac{\partial \rho }{\partial t}
+ \frac{\partial (\rho u) }{\partial x}
+ \frac{\partial (\rho v) }{\partial y}
+ \frac{\partial (\rho w) }{\partial z}
= 0,
$$

gdzie:\newline
w - wektor prędkości o rzutach u, v, w na osie współrzędnych x, y, z,\newline
p - ciśnienie,\newline
$\rho$ - gęstość,\newline
$\mu$ - współczynnik lepkości,\newline
X,Y,Z - rzuty wektora sił masowych K na osie współrzędnych,\newline
(pochodna substancjonalna) - pochodna substancjonalna,\newline

Równania są wyprowadzone na podstawie Newtonowskiego uogólnionego prawa tarcia.
Analiza płynów ściśliwych wymaga dodatkowo zastosowania równań uwzględniających ciśnienie, gęstość, temperaturę i przepływy energii.
Dla nieściśliwych izotermicznych płynów, równania przyjmują następującą formę:
(wstaw równanie)


\subsection{Metoda cząsteczkowa symulacji}
\subsubsection{Równania Naviera-Stokesa}
Przyjmijmy następującą formę równań dla cieczy nieściśliwych:\newline
(forma wiążąca)\newline
(zachowanie masy)\newline

Wykorzystanie metody cząsteczkowej pozwala na pewne uproszczenia, mianowicie:\newline
(1) Stała liczba cząsteczek zapewnia zachowanie masy wewnątrz systemu\newline
(2) wyrażenie (wstaw wyrażenie) może zostać zastąpione operatorem Stokesa(wstaw drugie). Ponieważ cząstki poruszają się wraz z płynem operator Stokesa pola prędkości jest pochodną czasu prędkości co znaczy, że wyrażenie (v*delta v) nie jest potrzebne.\newline
Oznacza to że równanie przyjmuje następującą formę:\newline
(równanie)\newline
Prawa strona równania składa się z trzech składowych:\newline
(-delta p) - odwrotność gradientu ciśnienia\newline
Odpowiada za siły wynikające z miejscowych różnic w ciśnieniu wewnątrz płynu.\newline
Współczynnik ten odpowiada za ruch miejscowy płynu podyktowany gradientem ciśnienia, powodując dążenie do wyrównania ciśnienia w ośrodku.\newline
(równanie) - siły zewnętrzne\newline
Odpowiadają siłom jednorodnym w symulowanej przestrzeni. Dotyczą one sił zewnętrznych jak grawitacja, a w przypadku symulacji objętości płynów poddawanych innym przyśpieszeniom wynikającym na przykład z ruchu ośrodka, również takowym.\newline
Ponieważ zastosowana wersja symulacji nie zakłada sił innych niż siła grawitacji równanie wygląda następująco:\newline
(rho g)\newline
(równanie) - siły lepkości\newline
Siły powodujące wygładzenie pola prędkości płynu spowodowane jego lepkością.\newline


\subsubsection{Metoda cząstek wygładzonych}
Metoda cząstek wygładzonych (ang. Smoothed Particle Hydrophysics - SPH) to bezsiatkowa metoda numeryczna stosowana do symulacji przepływów płynów i innych zjawisk ciągłych. Pierwotnie opracowana do symulacji zjawisk astrofizycznych aktualnie znajduje zastsowanie m.in. w mechanice płynów, inżynierii materiałowej, czy grafice komputerowej. Brak zastosowania siatki obliczeniowej jest dużą zaletą tej metody, pozwalającą na modelowanie znacznych deformacji, swobodnych powierzchni i zjawisk nieciągłych.\newline\newline
W metodzie tej ośrodek ciągły jest reprezentowany jako zbiór dyskretnych cząstek, które zawierają informację o lokalnych wartościach pól takich jak masa, gęstość, prędkość itp. Każda z cząstek oddziałowuje jedynie z cząstkami które znajdują się w jej bezpośrednim sąsiedztwie określonym przez promień wygładzania $h$. Wartości pól w danym punkcie przestrzeni są aproksymowane za pomocą funkcji wygładzającej, która dyktuje wagę z jaką uwzględniany jest wkład cząstek sąsiadujących w zależności od odległości.\newline\newline
Podstawą metody jest przybliżenie całki opisującej wielkość fizyczną w postaci sumy po cząstkach:\newline
$$
A(r) \approx \sum_{j}A_j \frac{m_j}{\rho_j}W(r-r_j,h),
$$
Gdzie:\newline
A - dowolna wielkość fizyczna,\newline
m - masa cząstki,\newline
$\rho$ - gęstość w punkcie cząstki,\newline
W - funkcja wygładzająca\newline
r - odległość od cząstki,\newline
h - odległość oddziaływania,\newline\newline
Równania opisujące zmiany stanu cząstek, takie jak równania Naviera-Stokesa są w metodzie SPH zapisywane w postaci dyskretnej, w oparciu o tą właśnie metodę obliczania wartości fizycznej w punkcie.\newline
Do zalet metody SPH należy prostota jej implementacji, łatwość zrównoleglenia oraz możliwość modelowania zjawisk z ruchomymi granicami.\newline
Głównymi wadami są natomiast wysoka złożoność obliczeniowa spowodowana koniecznością wyszukiwania cząstek sąsiadujących i problemy wynikające z trudności zachowania stabilności numerycznej oraz możliwa niska dokładność przy nieodpowiednich parametrach.\newline
Dlatego też metoda SPH wymaga zastosowania technik optymalizacyjnych i doboru odpowiednich parametrów symulacji.\newline


\subsubsection{Ciśnienie}
\subsubsection{Gęstość}
\subsubsection{Lepkość}
\subsubsection{Funkcje wygładzania}
Metoda wygładzonych cząstek opiera się na funkcjach wygładzania. Są to funkcje określające wzajemny wpływ cząstek w zależności od odległości.\newline
Funkcje wygładzania można traktować jako funkcje jednej zmiennej 
W(r) = r>h lub ...\newline
gdzie:\newline
r - odległość analizowanej cząstki od cząstki sąsiadującej\newline
h - maksymalna odległość interakcji\newline
Jest to forma zakładająca stałą maksymalną odległość interakcji, co powoduje ograniczenia w dowolności konfiguracji symulacji.\newline
Alternatywną i zazwyczaj stosowaną formą jest wyrażenie funkcji wygładzania jako funkcji dwóch zmiennych

\subsection{Odnajdywanie siłowe}
\subsection{Odnajdywanie skrótem}
\subsection{Metody drzewiaste}

\section{Obliczenia wydajności}
\section{Symulacja}
\section{Wyniki badań}
\section{Podsumowanie}
\section{Wnioski}
\section{Bibliografia}

\end{document}
