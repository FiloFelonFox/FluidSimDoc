\documentclass{article}

% Language setting
% Replace `english' with e.g. `spanish' to change the document language
\usepackage[polish]{babel}

% Set page size and margins
% Replace `letterpaper' with `a4paper' for UK/EU standard size
\usepackage[letterpaper,top=2cm,bottom=2cm,left=3cm,right=3cm,marginparwidth=1.75cm]{geometry}

% Useful packages
\usepackage{amsmath}
\usepackage{graphicx}
\usepackage[colorlinks=true, allcolors=blue]{hyperref}

\title{Praca inżynierska}
\author{ehhhhhhhhh}

\begin{document}
\newpage
\tableofcontents

\section{Wstęp}

\section{Cel i zakres}
Celem pracy jest stworzenie symulacji cząsteczkowej płynu opartej o równania Naviera-Stokesa oraz zbadanie wydajności zaproponowanej metody optymalizacji.

\section{Wstęp teoretyczny}
\subsection{Równania Naviera-Stokesa}
Równania Naviera–Stokesa stanowią podstawowy zestaw równań opisujących ruch płynów, zarówno cieczy, jak i gazów. Ich sformułowanie opiera się na zasadzie zachowania pędu, zgodnie z którą zmiany pędu elementarnej objętości płynu wynikają z oddziaływania sił działających na ten element. Równania zakładają że zmiany zależą od sił lepkości, sił wynikających z ciśnienia oraz sił masowe, takich jak siła grawitacji.\newline

Równania zapisane są w postaci równań różniczkowych cząstkowych i do ich rozwiązania niezbędne jest zastosowanie narzędzi rachunku różniczkowego i całkowego. Przez wzgląd na swoją nieliniowość oraz sprzężenie pomiędzy składowymi prędkości, ich rozwiązanie stanowi istotne wyzwanie obliczeniowe, co powoduje, że uzyskanie rozwiązań drogą analityczną jest realistyczne jedynie w najprostszych przypadkach, takich jak ustalony, laminarny i nieturbulentny przepływ w geometriach prostych.\newline

W przypadkach bardziej złożonych zagadnień, w szczególności przepływów niestacjonarnych, trójwymiarowych lub turbulentnych, stosowanych m.in. w obliczeniach aerodynamicznych i inżynierskich, konieczne jest wykorzystanie metod numerycznych. Metody te, implementowane w narzędziach obliczeniowej mechaniki płynów (CFD), umożliwiają przybliżone rozwiązanie równań Naviera–Stokesa i analizę rzeczywistych zjawisk przepływowych, które nie są możliwe do opisania metodami analitycznymi.\newline
Zapis równań Naviera-Stokesa w kartezjańskim układzie współrzędnych wygląda następująco:

$$ 
\rho 
\frac{du }{dt }
= X 
- \frac{\partial  p }{\partial x }
+ \frac \partial {\partial  x }
\left [ 
    \mu \left ( 
        2 \frac{\partial u }{\partial  x }
        - { \frac{2}{3} }  \mathop{\rm div}  \mathbf w 
    \right )  
\right ] 
+
$$

$$ 
+ 
\frac \partial {\partial  y }
\left [ 
    \mu \left ( 
        \frac{\partial u }{\partial  y }
        + \frac{\partial v }{\partial  x }
    \right )  
\right ]
+ 
\frac \partial {\partial  z }
\left [ 
    \mu \left ( 
        \frac{\partial w }{\partial  x }
        + \frac{\partial u }{\partial  z }
    \right )  
\right ],
$$

$$ 
\rho 
\frac{dv }{dt }
= Y
- \frac{\partial  p }{\partial y }
+ \frac \partial {\partial  y }
\left [ 
    \mu \left ( 
        2 \frac{\partial v }{\partial  y }
        - { \frac{2}{3} }  \mathop{\rm div}  \mathbf w 
    \right )  
\right ] 
+
$$

$$ 
+ 
\frac \partial {\partial  z }
\left [ 
    \mu \left ( 
        \frac{\partial v }{\partial  z }
        + \frac{\partial w }{\partial  y }
    \right )  
\right ]
+ 
\frac \partial {\partial  x }
\left [ 
    \mu \left ( 
        \frac{\partial u }{\partial  y }
        + \frac{\partial v }{\partial  x }
    \right )  
\right ],
$$

$$ 
\rho 
\frac{dw }{dt }
= Z 
- \frac{\partial  p }{\partial z }
+ \frac \partial {\partial  z }
\left [ 
    \mu \left ( 
        2 \frac{\partial w }{\partial  z }
        - { \frac{2}{3} }  \mathop{\rm div}  \mathbf w 
    \right )  
\right ] 
+
$$

$$ 
+ 
\frac \partial {\partial  x }
\left [ 
    \mu \left ( 
        \frac{\partial w }{\partial  x }
        + \frac{\partial u }{\partial  z }
    \right )  
\right ]
+ 
\frac \partial {\partial  y }
\left [ 
    \mu \left ( 
        \frac{\partial v }{\partial  z }
        + \frac{\partial w }{\partial  y }
    \right )  
\right ],
$$

$$ 
\frac{\partial \rho }{\partial t}
+ \frac{\partial (\rho u) }{\partial x}
+ \frac{\partial (\rho v) }{\partial y}
+ \frac{\partial (\rho w) }{\partial z}
= 0,
$$

gdzie:\newline
w - wektor prędkości o rzutach u, v, w na osie współrzędnych x, y, z,\newline
p - ciśnienie,\newline
$\rho$ - gęstość,\newline
$\mu$ - współczynnik lepkości,\newline
X,Y,Z - rzuty wektora sił masowych K na osie współrzędnych,\newline
(pochodna substancjonalna) - pochodna substancjonalna,\newline

Równania są wyprowadzone na podstawie Newtonowskiego uogólnionego prawa tarcia.
Analiza płynów ściśliwych wymaga dodatkowo zastosowania równań uwzględniających ciśnienie, gęstość, temperaturę i przepływy energii.
Dla nieściśliwych izotermicznych płynów, równania przyjmują następującą formę:
(wstaw równanie)


\subsection{Metoda cząsteczkowa symulacji}
\subsubsection{Równania Naviera-Stokesa}
Przyjmijmy następującą formę równań dla cieczy nieściśliwych:
(forma wiążąca)
(zachowanie masy)

Wykorzystanie metody cząsteczkowej pozwala na pewne uproszczenia, mianowicie:\newline
(1) Stała liczba cząsteczek zapewnia zachowanie masy wewnątrz systemu\newline
(2) wyrażenie (wstaw wyrażenie) może zostać zastąpione operatorem Stokesa(wstaw drugie). Ponieważ cząstki poruszają się wraz z płynem operator Stokesa pola prędkości jest pochodną czasu prędkości co znaczy, że wyrażenie (v*delta v) nie jest potrzebne.\newline



\subsubsection{Modelowanie cząsteczkami}
\subsubsection{Ciśnienie}
\subsubsection{Gęstość}
\subsubsection{Lepkość}
\subsubsection{Funkcje wygładzania}

\subsection{Odnajdywanie z wykorzystaniem skrótów}


\section{Wyniki badań}
\subsection{Odnajdywanie siłowe}
\subsection{Odnajdywanie skrótem}
\section{Podsumowanie}
\section{Wnioski}
\section{Bibliografia}

\end{document}
