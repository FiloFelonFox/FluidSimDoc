\documentclass{article}

% Language setting
% Replace `english' with e.g. `spanish' to change the document language
\usepackage[polish]{babel}

% Set page size and margins
% Replace `letterpaper' with `a4paper' for UK/EU standard size
\usepackage[letterpaper,top=2cm,bottom=2cm,left=3cm,right=3cm,marginparwidth=1.75cm]{geometry}

% Useful packages
\usepackage{amsmath}
\usepackage{graphicx}
\usepackage[colorlinks=true, allcolors=blue]{hyperref}

\usepackage{algorithm}
\usepackage{algorithmic}

\title{Praca inżynierska}
\author{ehhhhhhhhh}

\begin{document}

\section{Wstęp}

\section{Cel i zakres}

\section{Wstęp teoretyczny}
\subsection{Równania Naviera-Stokesa}
\subsection{Metoda cząsteczkowa symulacji}
\subsubsection{Metoda cząstek wygładzonych}
\subsubsection{Funkcje wygładzania}
\subsubsection{Równania Naviera-Stokesa}
\subsubsection{Ciśnienie}
\subsubsection{Siły zewnętrzne}
\subsubsection{Lepkość}

\section{Metody odnajdywania cząstek}
\subsection{Metoda siłowa}
Opis\\
Pseudokod\\
Obliczenia złożoności\\
\subsection{Siatka jednorodna}
Opis wytworzenie\\
Pseudokod wytworzenie\\
Obliczenia złożoności wytworzenie\\
Opis odnajdywanie\\
Pseudokod odnajdywanie\\
Obliczenia złożoności odnajdywanie\\
\subsection{Haszowanie przestrzenne}
Opis przypisanie\\
Pseudokod przypisanie\\
Obliczenia złożoności przypisanie\\
Opis wytworzenie tablicy \\
Pseudokod wytworzenie tablicy\\
Obliczenia złożoności wytworzenie tablicy\\
Opis odnajdywanie\\
Pseudokod odnajdywanie\\
Obliczenia złożoności odnajdywanie\\
\subsection{Stały indeks}
Opis przypisanie\\
Pseudokod przypisanie\\
Obliczenia złożoności odnajdywanie\\
\subsection{Metody drzewiaste}

\section{Obliczenia wydajności}
\section{Symulacja}
\section{Wyniki badań}
\section{Podsumowanie}
\section{Wnioski}
\section{Bibliografia}

\end{document}
